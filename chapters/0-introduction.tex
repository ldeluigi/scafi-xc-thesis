% ! TeX root = ../thesis-main.tex
%----------------------------------------------------------------------------------------
\chapter{Introduction}
\label{chap:introduction}
%----------------------------------------------------------------------------------------
\acp{CAS} are a particular kind of situated, distributed systems where a collection of individuals, also called agents, exhibits a non-chaotic behavior with \textit{self-*} properties, such as self-organization, self-healing, and self-configuration\cite{macroprogramming-state-of-the-art}.
%
Many of the properties mentioned above cannot be reduced to or derived from an individual perspective on the behavior of one of the agents, but rather are \textit{emergent} from the complex network and dynamic interactions within the system and with the environment\cite{macroprogramming-state-of-the-art}.
%
\acp{CAS} can be considered a subset of \acp{MAS}, and the programming of their behavior as a whole can be referred to as \textit{Macroprogramming}\cite{macroprogramming-state-of-the-art}.

With the expected diffusion of large-scale cyber-physical systems, pushed by the already pervasive \ac{IoT} and edge computing trends\cite{scafi}, \acp{CAS} engineering is 



%--------------DELETE:--->
Write your intro here.
\sidenote{Add sidenotes in this way. They are named after the author of the thesis}

You can use acronyms that your defined previously,
such as \ac{IoT}.
%
If you use acronyms twice,
they will be written in full only once
(indeed, you can mention the \ac{IoT} now without it being fully explained).
%
In some cases, you may need a plural form of the acronym.
%
For instance,
that you are discussing \acp{VM},
you may need both \ac{VM} and \acp{VM}.

\paragraph{Structure of the Thesis}

\note{At the end, describe the structure of the paper}

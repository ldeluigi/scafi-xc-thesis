% ! TeX root = ../thesis-main.tex
%----------------------------------------------------------------------------------------
\chapter{Design}
\label{chap:design}
%----------------------------------------------------------------------------------------
The design of \this has been divided into three main phases.
%
In the first part, four design prototypes have been developed in order to choose the best user experience for three key users, the \textit{program developer}, the \textit{library developer}, and finally \textit{a new foundation researcher}, which represents a novelty in the use cases of an aggregate programming library.

In the second part, the final version of the \ac{DSL} has been designed, taking the best features from the prototypes and integrating them into a new core \ac{DSL} for \this.
%
An execution context for aggregate programs made with \this has been designed and called \quotes{engine}.

In the third and final part, the design process concerned the execution engine and the simulator.

The first and second part cover requirements \textit{F.1}, \textit{F.3}, \textit{F.6}, \textit{F.7}, and \textit{T.1}, while the third covers requirement \textit{F.5} from \cref{chap:analysis->sec:requirement-analysis}.

\section{Designing a scalable internal Domain Specific Language} \label{chap:design->sec:dsl}

The process of designing a new core \ac{DSL} for \this has been carried out through rapid prototyping of four competing designs of different \acp{DSL}, each coming with a set of advantages and disadvantages, highlighted using code snippets for every key user.
%
Prototipation has been necessary to explore the design space and to understand the trade-offs between different design choices, as well as to test in practice the interactions of different combinations of language features.
%
For each of the four prototypes, a brief description of the design choices and the programming experience is provided, followed by the final design described in \cref{chap:design->sec:final-dsl}.
%
Each prototype is named after the Scala 3 feature it is mainly based on, and each aims to separate the definition of the syntax from the definition of the semantics, and separate the definition of the semantics from the actual implementation.
%
This way, more than one semantic can be defined for the same syntax, and more than one implementation can be defined for the same semantic, allowing for more flexible customization and composition.
%
For the scope of \this, \ac{XC}\cite{xc} is the only semantics considered for implementation, but the design should be flexible enough to allow for future \quotes{calculi} to be implemented as alternative foundations.
%
In practice, there is a very subtle difference between the semantics and the syntax of the same formal calculus.
%
In the case of \ac{XC}, the syntax and the semantics both consist of two methods, \texttt{branch} and \texttt{exchange}.
%
The difference lies in the role of the two methods in the two contexts: in the syntax, they are constructs that the programmer uses to write programs, and should be treated as an \ac{API} at all effects, while in the semantics, they are constructs that the programmer uses to implement all the supported syntaxes \ac{API}, meaning they are not part of an \ac{API} used by aggregate programs and are focused on being as complete and simple to implement as possible.
%
For instance, the \texttt{exchange} method of the \texttt{ExchangeCalculusSyntax} trait is focused on its usage experience, attempting to mimick the syntax provided in the paper, meanwhile the \texttt{xcexchange} method of the \texttt{ExchangeCalculusSemantics} has \texttt{protected} visibility and its signature provides only the most complete and expressive version of the \texttt{exchange} primitive presented in the paper\cite{xc}.

Each of the four prototypes retains the separation between syntax definitions, semantic definitions, syntax support declaration, and semantics implementations, and considers the programmer experience of all the three key users.
%
For brevity, only the most relevant advantages and disadvantages are highlighted, and the full code of the prototypes is available in the project repository git history, under the git tag \texttt{experiments}\footnote{\url{https://github.com/ldeluigi/scafi-xc/tree/experiments}}, after which the code got removed to avoid confusion with the final design.


\subsection{Prototype 1: Extension Methods} \label{chap:design->sec:dsl->subsec:prototype-1-extension-methods}

In this design, an \texttt{AggregateFoundation} base trait defines a common syntax for all the aggregate programming foundations, such as the existence of a type called \texttt{AggregateValue[T]} that represents a collection of values coming from neighboring devices, including self.
%
The \texttt{AggregateFoundation} trait also defines a set of \textit{abstract extension methods} that provide basic functionalities for aggregate values, such as \textit{lifting} for composition and mapping, \textit{folding} for reduction, methods for retrieving the value for the current device or exclude the current device, mimicking the functionalities of the \texttt{foldhood} and \texttt{foldhoodPlus} of ScaFi, as shown in \cref{lst:prototype-1-aggregate-foundation}.
%
In the example, the source of the \texttt{Liftable} and \texttt{Foldable} type classes is included for completeness, but they are not part of the \texttt{AggregateFoundation} source file as they are located in the \texttt{commons} module.

\lstinputlisting[float, language=Scala, caption={Prototype 1 - Aggregate Foundation and helper type classes.}, label={lst:prototype-1-aggregate-foundation}]{listings/prototype-1-aggregate-foundation.scala}

By defining an abstract type member \texttt{AggregateValue[T]}, semantics like \ac{XC} can override it to model any kind of specific interface, such as \textit{NValues}, on top of which they can provide any additional behavior and syntax, following the pattern of \textit{family polimorphism}.
%
Implicitly, this design abandons the \quotes{field-transparent} semantics of the \texttt{foooldhood*} methods of ScaFi in favor of having explicit \textit{field} types, similarly to FCPP (\cref{chap:state-of-the-art->sec:fcpp}) and the \ac{XC} \ac{DSL} experiment (\cref{chap:state-of-the-art->sec:xc-experiment}).
%
Nevertheless, a semantics design that replicates that lost feature can still be implemented with an extension of \texttt{AggregateFoundation} that provides a \texttt{foldhood} construct that works the same as the \texttt{foldhood*} methods of ScaFi.

\textit{Aggregate Semantics} such as \texttt{ExchangeCalculusSemantics} are defined as a trait that extends \texttt{AggregateFoundation}.
%
A semantics trait can provide a concrete type for the abstract type member \texttt{AggregateValue[T]}, even if the type refers to a trait and not a concrete implementation.

For instance, the \texttt{ExchangeCalculusSemantics} provides a concrete implementation for \texttt{AggregateValue[T]} corresponding to \texttt{NValues[ID, T]}, where \texttt{ID} is an abstract type member for device identifiers.
%
Additionally, the semantics provides an abstract given instance of \texttt{CanEqual[ID, ID]} to provide equality comparisons between device identifiers, as well as the core \ac{XC} constructs: \texttt{xcexchange} and \texttt{xcbranch}, corresponding to the \texttt{exchange} primitive and the domain branching behavior of \ac{XC}.
%
These core constructs of the semantics are protected in visibility because they are meant to be invoked only through a facade that corresponds to one or more syntax of a \textit{calculus}, whereas abstract given instances are public as they are meant to be available to libraries that depend on a specific semantics.
%
The syntaxes for both \ac{FC} and \ac{XC} are shown in \cref{lst:prototype-1-syntaxes}, and the \ac{XC} semantics is shown in \cref{lst:prototype-1-exchange-calculus-semantics}.
%
Finally, \cref{lst:prototype-1-syntaxes-impl} shows the implementation of the syntaxes in terms of the \ac{XC} semantics.

One notable feature of having a facade defined using extension methods is that the compatibility layer between a semantic and a syntax is provided through the implementation of a given instance, much like a written proof that a syntax can be obtained extending a given semantics.
%
The proof can \texttt{summon} other given instances of supported syntax in order to define a proof dependent on another proof, as shown in \cref{lst:prototype-1-syntaxes-impl}.
%
Another significant feature is the possibility to import dependencies and preferred syntax/facade with an \texttt{import} statement at the beginning of the file, instead of having to mix in traits like in ScaFi.
%
Additionally, this design allows implementing a new library by simply writing new extension methods of a generic language \texttt{L} that extends \texttt{AggregateFoundation} and the required other syntaxes, as shown in \cref{lst:prototype-1-usage-lib}.
%
This allows libraries to be singleton objects, imported where needed with a top-level \texttt{import} statement.
%
One hidden feature of this design is the possibility to hide, by default, transitive library dependencies, something that ScaFi could not allow because it used a mixin composition where every library was a trait that mixed in other traits, thus exposing all the dependencies of the mixed-in traits.

\lstinputlisting[float, language=Scala, caption={Prototype 1 - Syntax definitions.}, label={lst:prototype-1-syntaxes}]{listings/prototype-1-syntaxes.scala}
\lstinputlisting[float, language=Scala, caption={Prototype 1 - Exchange Calculus Semantics.}, label={lst:prototype-1-exchange-calculus-semantics}]{listings/prototype-1-exchange-calculus-semantics.scala}
\lstinputlisting[float, language=Scala, caption={Prototype 1 - Syntax implementations in terms of the exchange semantics.}, label={lst:prototype-1-syntaxes-impl}]{listings/prototype-1-syntaxes-impl.scala}

Even though this design proves to be very flexible and extensible, it has a few drawbacks.
%
The most important one is the impossibility of invoking an extension method without the \quotes{\texttt{.}}, having at best to invoke constructs on \texttt{this}, as in \cref{lst:prototype-1-usage}.
%
The next prototypes focus on overcoming this limitation at any cost, even if it means losing some of the flexibility and clarity of the design, in order to provide empirical evidence of the trade-offs between the different design choices.

\lstinputlisting[float, language=Scala, caption={Prototype 1 - Example usage by an aggregate program developer.}, label={lst:prototype-1-usage}]{listings/prototype-1-usage.scala}
\lstinputlisting[float, language=Scala, caption={Prototype 1 - Example usage by a library developer.}, label={lst:prototype-1-usage-lib}]{listings/prototype-1-usage-lib.scala}


\subsection{Prototype 2: Context parameter in constructors} \label{chap:design->sec:dsl->subsec:prototype-2-implicit-parameter-in-constructors}

Even though the second prototype is based on the use of a context parameter that passes a semantics instance down to every construct invocation, the folding and lifting functionalities are provided through abstract given instances of type classes declaring extension methods, as per prototype 1.
%
Libraries, as well as core syntaxes, are defined as classes and traits, respectively, that take a context parameter of type \texttt{L}, short for \texttt{Language}, that must be a subtype of \texttt{AggregateFoundation}, as shown in \cref{lst:prototype-2-syntax}.
%
Libraries dependent on other libraries must either instantiate their dependencies or require a context parameter that provides them, as shown in \cref{lst:prototype-2-usage-lib}.
%
This approach has the side effect that the type member \texttt{AggregateValue} present in the \texttt{AggregateFoundation} is seen as a different type for every dependent method of every library, thus making passing an aggregate value from a library method to another impossible.
%
This forced each semantics, library, and syntax to have a generic type constructor parameter \texttt{AV[\_]} that generalizes and uniformizes the type of the aggregate value for all the dependencies, making them compatible again.
%
\cref{lst:prototype-2-usage-lib} also shows how this design, with the additional cost of a static \texttt{import} for dependencies methods and given instance, allows to invoke constructs such as \texttt{branch} or \texttt{distanceTo} with having to use the \texttt{.} operator and the same applies to user programs.

\lstinputlisting[float, language=Scala, caption={Prototype 2 - \ac{FC} syntax definition.}, label={lst:prototype-2-syntax}]{listings/prototype-2-syntax.scala}
\lstinputlisting[float, language=Scala, caption={Prototype 2 - Example usage by a library developer.}, label={lst:prototype-2-usage-lib}]{listings/prototype-2-usage-lib.scala}

Having to explicitly import the methods from dependencies inside every library of programs, added to the necessity of instantiating libraries, makes this design very cumbersome to use.
%
The next prototype attempts to improve the usability by returning to libraries defined as singleton objects, having each construct take the context parameter that were in the library class constructor in this prototype.


\subsection{Prototype 3: Implicit parameter in methods} \label{chap:design->sec:dsl->subsec:prototype-3-implicit-parameter-in-methods}

This prototype design shares some similarities with the previous one, but it has a different approach to the problem of passing the semantics instance to the constructs, and to dealing with library dependencies.
%
In this design, libraries are singleton objects, where each method takes a context parameter for the semantics and a context parameter for every syntax needed as a dependency, as shown in \cref{lst:prototype-3-usage-lib}.
%
Alternatively, a library method can instantiate its syntax dependencies and only take a context parameter for the semantics, in a similar way to the previous prototype.
%
Even though the issue of library dependencies is solved, syntax dependencies are still cumbersome to use and still require static imports inside the body of the methods to invoke the constructs without the \texttt{.} operator.
%
In addition to that, \texttt{AggregateFoundation} still needs the generic aggregate value type constructor parameter to make syntax dependencies compatible with each other.

\lstinputlisting[float, language=Scala, caption={Prototype 3 - Example usage by a library developer.}, label={lst:prototype-3-usage-lib}]{listings/prototype-3-usage-lib.scala}

The last prototype represents a return to the original mixin-oriented design used in ScaFi and its purpose is closer to a comparison baseline rather than a design alternative, but it is still useful for the last design phase where the best features of all the prototypes will be cherry-picked and combined.

\subsection{Prototype 4: Mixin composition} \label{chap:design->sec:dsl->subsec:prototype-4-mixin-composition}

In this design, the \texttt{AggregateFoundation} trait is the same as in prototype 1 (see \cref{lst:prototype-1-aggregate-foundation}), without the need for the generic type constructor for aggregate values, because syntaxes, semantics, and the foundation are meant to become part of the same type hierarchy, and a type member for aggregate values will be the same for all the mixed-in traits.
%
For instance, given a semantics such as \texttt{ExchangeCalculusSemantics}, giving proof for the support for a syntax means to define a trait to be mixed in with the semantics, that implements the syntax in terms of the semantics.
%
Libraries are defined, like in ScaFi, with mixin traits that declare their dependencies using self-type annotations, as shown in \cref{lst:prototype-4-usage-lib}.

The main advantage of this design is the possibility to invoke constructs without the \texttt{.} operator, as shown in \cref{lst:prototype-4-usage-lib}, while the main disadvantage is the need to inherit from all the transitive dependencies together in the program class, and having to honor a global construct naming consistency both in all the libraries and in the semantics implementation.

\lstinputlisting[float, language=Scala, caption={Prototype 4 - Example usage by a library developer.}, label={lst:prototype-4-usage-lib}]{listings/prototype-4-usage-lib.scala}

\section{Final design of the core DSL} \label{chap:design->sec:final-dsl}

Taking inspiration from the best features of all the prototypes, the final design was developed and showcased with a presentation in front of the research group, which provided positive feedback on the resulting user experience.
%
The final design consists of an \texttt{AggregateFoundation} similar to prototypes 1 and 4, with core syntaxes and libraries defined as traits for a mixin composition.
%
The twist is that libraries are instead defined as singleton objects, able to be imported with a top-level \texttt{import} statement, without having visibility on transitive dependencies and without having to mix them in together with the semantics in the program class.
%
The disadvantage of this design would have been the different invocation syntax for library constructs versus core syntax constructs such as \texttt{nbr} and \texttt{rep}.
%
This disadvantage has been overcome by defining a facade library for every core syntax, hiding the \quotes{\texttt{language.}} prefix necessary for invoking core syntax constructs, with the small cost of having to write a facade library for every future syntax developed by researchers.
%
An \ac{UML} diagram of the final model for the foundation is shown in \cref{fig:final-design-foundation-diagram}.

\begin{figure}
    \centering
    \caption{Final design: \ac{UML} diagram of the \texttt{AggregateFoundation}.}
    \label{fig:final-design-foundation-diagram}
    \bigskip
    \resizebox{\linewidth}{!}{
        \input{diagrams/final-design-foundation/Final Design Foundation.latex}
    }
\end{figure}

\Cref{fig:final-design-exchange-calculus-semantics-diagram} shows the \ac{UML} diagram of the \texttt{ExchangeCalculusSemantics} mixin composition, which is the only semantics implemented for the scope of this project.

\begin{figure}
    \centering
    \caption{Final design: \ac{UML} diagram of the \texttt{ExchangeCalculusSemantics} mixin composition.}
    \label{fig:final-design-exchange-calculus-semantics-diagram}
    \bigskip
    \resizebox{\linewidth}{!}{
        \input{diagrams/final-design-semantics/Final Design Semantics.latex}
    }
\end{figure}

Thanks to this design, the gradient construct, also known as \texttt{distanceTo}, can be defined to work with any aggregate semantics that supports the field calculus syntax, and thanks to Scala context bounds it can also be defined to work with any numeric type that supports an upper bound, such as \texttt{Double}.
%
The resulting code for the \texttt{distanceTo} construct is shown in \cref{lst:distance-to}, which resembles the syntax of the prototype \ac{DSL} of \cref{lst:gradient-distance-xc-scala2-dsl}.
%
The provided snippet demonstrates how this design effectively achieves the goals pursued by the other design prototypes:
\begin{itemize}
    \item invocation of library and core constructs without the \texttt{.} operator;
    \item declaration of library dependencies with top-level imports, that hide transitive dependencies;
    \item generic definition of the constructs for high reusability;
    \item clean declaration of the core syntax dependencies, using a \quotes{\texttt{language}} context parameter and intersection types;
    \item possibility to resolve naming conflicts with import aliases;
    \item reuse of all the libraries and programs dependent on a set of syntaxes by re-implementing them through the next generation of aggregate programming calculi;
\end{itemize}

\lstinputlisting[float, language=Scala, caption={Final design - \texttt{distanceTo} implementation in the \texttt{GradientLibrary}.}, label={lst:distance-to}]{listings/distance-to.scala}

\subsection{Design of the XC operational semantics} \label{chap:design->sec:final-dsl->subsec:exchange-calculus-semantics-design}

By abstracting common features of aggregate programming languages such as \ac{FC} and \ac{XC} into the \texttt{AggregateFoundation} trait, the \texttt{ExchangeCalculusSemantics} can focus on the peculiarities of the \ac{XC} semantics, such as NValues and the \texttt{exchange} primitive.
%
\texttt{AggregateFoundation} provides an abstract definition of an \texttt{AggregateValue} type, with the only feature to be iterable.
%
The reason is that whatever the next aggregate calculus semantics will be, it is expected to provide some kind of notion of field, which should allow iterating over values from neighbors, including self.
%
Additionally, \texttt{AggregateFoundation} provides the \ac{API} to exclude the value for self from an \texttt{AggregateValue}, as well as to retrieve the value for self only, using extensions method inspired by the \ac{XC} \ac{DSL} experiment of \cref{chap:state-of-the-art->sec:xc-experiment}.
%
Referencing self has an important role in aggregate programs, and it was put here to provide all libraries and programs with that feature.
%
Finally, \texttt{AggregateFoundation} provides the means to combine and map aggregate values with the \texttt{lift} operator and the \texttt{map} extension method.
%
This is a necessary difference from the formal calculi, where aggregate values such as NValues or fields allow to be treated as their underlying generic type, and are transparent when combined or mapped as if they were local values.
%
For example, in \ac{XC}, an expression $e$ written to work with local values of type \texttt{T} can be used with an aggregate value of type \texttt{NValues[T]} without any modification.
%
In Scala, instead, this is not possible, and operations on local values need to be \textit{lifted} to work on aggregate values too.
%
Unary operations have to be lifted too and work as the mapping function \texttt{f: A => B} for the \texttt{map} extension method available on an \texttt{AggregateValue[A]}.
%
As mentioned in the previous section, abstracting common features into a common foundation for all the semantics allows the reuse of all the libraries and programs depending on those features, while also retaining the possibility to re-implement differently in a new aggregate calculus semantics.

To allow the definition, in the future, of an aggregate calculus without explicit device identifiers exposed in the \ac{API}, all the features related to explicit device identifiers in the calculus have been grouped and modeled as an optional mixin called \texttt{DeviceAwareAggregateFoundation}.
%
The mixin defines an abstract type member \texttt{DeviceId} and the means to compare them with \texttt{==}, that is a given instance of \texttt{CanEqual[DeviceId, DeviceId]}.
%
In addition to that, it provides two abstract methods, one called \texttt{self} that returns the device identifier of the current device, and another called \texttt{device} that returns an aggregate value of device identifiers, including self, which is always known thanks to the network and it doesn't need to be computed with, for example, a \texttt{nbr(self)} invocation.

Therefore, the \texttt{ExchangeCalculusSemantics} is left only with the definitions of its specific features, which are NValues additional operation on aggregate values, automatic conversion from local values and NValues, the core constructs \texttt{exchange}, here called \texttt{xc}, and \texttt{branch}, called \texttt{br}, to avoid conflicts with their counterparts defined in the implemented syntaxes.
%
The signature of the \texttt{xc} method has been simplified to simplify its implementation, written only in its complete form with both the \texttt{return} and \texttt{send} values explicitly passed as a couple, whereas the \texttt{exchange} method of the syntax allows different call signatures to imitate the syntax of the paper\cite{xc}, as shown in \cref{lst:exchange-syntax}.

\lstinputlisting[float, language=Scala, caption={Supported syntaxes for invoking the \texttt{exchange} primitive.}, label={lst:exchange-syntax}]{listings/exchange-syntax.scala}

The concrete implementation of the \ac{XC} operational semantics is discussed partly in the next section, and partly in \cref{chap:implementation->sec:xc-ops}.

\subsection{The Engine} \label{chap:design->sec:final-dsl->subsec:engine}

An \texttt{Engine} has been designed to be able to execute aggregate programs that use an instance of the \ac{XC} semantics trait as a context parameter.
%
The \texttt{Engine} offers a method called \texttt{cycle} which implements all the steps to be executed in a single round of an aggregate program.
%
The steps can be summarized into the following:
\begin{enumerate}
    \item instantiation of the semantics, whose implementation is called \texttt{Context}, using information coming from the \texttt{Network}, such as inbound messages and the device identifier;
    \item execution of the aggregate program, which is a function that takes the \texttt{Context} as a context parameter and returns a \texttt{Result}, which is the result of the evaluation of the aggregate program;
    \item collection of the \texttt{Export}, which is a bundle containing all the outbound messages;
    \item sending of the \texttt{Export} to the \texttt{Network}, which will deliver the messages to the intended recipients.
\end{enumerate}
%
A \texttt{Context} is defined as an interface that takes inbound messages as input, called \texttt{Import}, gets altered by the aggregate program round of execution, and produces outbound messages as output, called \texttt{Export}.
%
More information about the implementation of a \texttt{Context} for the \ac{XC} semantics can be found in \cref{chap:implementation->sec:xc-ops}.
%
An \texttt{Import} is defined as an alias for a map from device identifiers to \textit{value trees}.
%
A \texttt{ValueTree} is an \ac{ADT} containing the values exchanged between devices coupled with their path in the \ac{AST} of the aggregate program, as described in the \ac{XC} paper\cite{xc}.
%
Here a \texttt{Path} is defined as an alias for a \texttt{List} of generic tokens so that every implementation of a semantic can control which type of token defines a location inside the \ac{AST} of the aggregate program.
%
An \texttt{Export} instead is defined as an alias for a \texttt{MapWithDefault}, because it can send a dedicated message to a known neighbor and a default message to every other new neighbor of the device.
%
The \texttt{Network} is responsible for properly dispatching the messages to the intended recipients and is pictured with \ac{UML} in \cref{fig:engine-network-diagram}.

\begin{figure}
    \centering
    \caption{The engine: \ac{UML} diagram of the \texttt{Network} interface.}
    \label{fig:engine-network-diagram}
    \bigskip
    \resizebox{\linewidth}{!}{
        \input{diagrams/network/Network.latex}
    }
\end{figure}

In summary, the \texttt{Engine}, uses a \texttt{ContextFactory} to repeatedly instantiate a new \texttt{Context} for every cycle of the aggregate program execution, executes the program against the context, and finally sends the result to the network.
%
An \ac{UML} diagram of the \texttt{Context} and the \texttt{Engine} can be found in \cref{fig:engine-diagram}.

\begin{figure}
    \centering
    \caption{The engine: \ac{UML} diagram of the \texttt{Context} interface and the \texttt{Engine} class.}
    \label{fig:engine-diagram}
    \bigskip
    \resizebox{\linewidth}{!}{
        \input{diagrams/engine/Engine.latex}
    }
\end{figure}


\section{Network-based sensors} \label{chap:design->sec:network-based-sensors}

In general, sensors and actuators are not part of the \ac{DSL} of \this.
%
In ScaFi, sensors are modeled as a key-value dictionary, from \texttt{String} to \texttt{Any}, that casts the value to the expected type.
%
To improve the type safety and to have less error-prone sensor access, in this design sensor and actuator design and implementation are left to the user, that can provide them in three main ways:
\begin{enumerate}
    \item by implementing an external library that interacts with the sensor and actuator hardware, which gets invoked by the aggregate program;
    \item by extending the \texttt{Context} implementation with additional, type-safe sensor and actuator methods, invoked in the aggregate program in the form \texttt{ctx.sensorName()} or \texttt{summon[ContextImpl].sensorName()};
    \item by both extending the \texttt{Context} and implementing an external library as its facade, to provide an \ac{API} with static method signatures that can be invoked in the aggregate program, in line with the style of the \ac{DSL}.
\end{enumerate}
%
This design choice has been made to keep the \ac{DSL} as simple as possible and to allow the user to choose the best way to interact with the sensor and actuator hardware, as well as to allow the user to choose the best way to model the sensor and actuator data, which can be very different from one application to another.
%
Nevertheless, network-based sensors, which are sensors whose measured values differ for every visible neighbor, represent a special case.
%
These sensors must either be implemented following the second or the third way of the list above, because a measurement must return an \texttt{AggregateValue[T]} where \texttt{T} is the type of the measurement and the aggregate value contains a measurement for every visible neighbor, including self.

The standard library included in the \texttt{core} module provides a network-based sensor called \texttt{DistanceSensor}, due to its importance in common aggregate programs.
%
In ScaFi, the corresponding construct is called \texttt{nbrRange}.
%
If an aggregate context implements the \texttt{DistanceSensor[N: Numeric]} trait, \texttt{senseDistance: N} is available to be invoked in the aggregate program, and it returns an \texttt{AggregateValue[N]} containing the distance to every visible neighbor, including self.
%
The availability of the distance sensor in the context enables the invocation of library constructs based on that, such as the \texttt{sensorDistanceTo} of the \texttt{GradientLibrary} which uses the sensors as metric for the distance to the neighbors.
%
The distance sensor is generic on the type of the measurement, as long as it is a numeric type, to allow for different configurations, such as measuring with floats, integers, or custom types that provide a \texttt{Numeric} given instance.


\section{The simulator}

The simulator module serves both as a tool for developers to test their programs in a controlled environment and as a fundamental component for the acceptance tests.
%
For the scope of this project, the simulator has been designed to be as simple as possible, providing a minimal set of functionalities that are enough to test the \ac{DSL} and the libraries developed for it.
%
As a result, the simulator is deterministic, with discrete time, and models neighborhoods as a map from device identifiers to a set of device identifiers.
%
Nevertheless, the simulator implements basic real-world network phenomena such as message loss and delay, as well as customizable message retention time and device reboot/failures.
%
Inside tests, a \textit{deterministic} simulator allows control of every aspect of the aforementioned features through the use of policies, implementing the strategy pattern.
%
In addition to that, in manual tests, a \textit{random} simulator could be used.
%
The random simulator allows the generation of randomized device networks and randomized policies to simulate an environment closer to the internet, following a set of given parameters for the probability distributions used in the implementation.
%
This is particularly useful to test the self-healing, self-organizing properties of aggregate programs.
%
Tests can be reproduced deterministically even in the random simulator thanks to the \texttt{seed} parameter that controls the generation of pseudo-random numbers.
%
The resulting \ac{UML} diagram of the simulator module design is shown in \cref{fig:simulator-uml}.

\begin{figure}
    \centering
    \caption{\ac{UML} diagram of the \texttt{simulator} module.}
    \label{fig:simulator-uml}
    \bigskip
    \resizebox{\linewidth}{!}{
        \input{diagrams/simulator/Simulator.latex}
    }
\end{figure}

% ! TeX root = ../thesis-main.tex
%----------------------------------------------------------------------------------------
\chapter{State of the art}
\label{chap:state-of-the-art}
%----------------------------------------------------------------------------------------

In the field of aggregate programming~\cite{aggregate-programming}, multiple frameworks, tools, and experiments have been developed and made available as public resources to support a wide variety of use cases, using different languages and design approaches.
%
Among the state-of-the-art tools in this field are \textit{ScaFi}\footnote{\url{https://github.com/scafi/scafi}}~\cite{scafi}, \textit{Protelis}\footnote{\url{https://github.com/Protelis/Protelis}}~\cite{protelis}, and \textit{FCPP}\footnote{\url{https://github.com/fcpp/fcpp}}~\cite{fcpp} which have been briefly described in their fundamental characteristics in this chapter, with a particular emphasis on ScaFi, of which \this is a redesign and reimplementation.
%
Each of the mentioned libraries is backed by a robust, coherent theoretical foundation, that provides consistency and guarantees the emergence of global properties in derived \acp{CAS}.
%
The theoretical framework that serves as the basis for all cited implementations is the \ac{FC}~\cite{fc}, specifically its higher-order version, the \ac{HFC}~\cite{hofc}.
%
\ac{FC}, as well as its variants, is a type-safe, formal language for aggregate programming~\cite{fc, from-dc-to-fc-and-ap} presented with its operational and denotational semantics, respectively describing the local and global interpretation of field expressions~\cite{from-dc-to-fc-and-ap}.
%
From a developer perspective, the key aspect of \ac{FC} is the possibility of focusing on the denotational semantics of field constructs, abstracting away from the local interpretation of expressions and implementation of the constructs.
%
In recent years, a new formal language called \ac{XC}~\cite{xc} has been developed, which is a promising evolution of \ac{FC}.
%
\ac{XC}~\cite{xc} has the potential to supersede \ac{FC} entirely since it is a simpler yet more expressive language that can be used to implement all the \ac{FC} constructs while retaining their original semantics.
%
\this and FCPP in its current version are based on this newer formal language, further described in \Cref{chap:background}, where \ac{FC} and \ac{XC} are briefly compared.
%
Some additional experiments with the implementation of \ac{XC} already exist, such as \textit{imperative-xc}\footnote{\url{https://github.com/cric96/imperative-xc}} and \textit{XC: Scala DSL Implementation}\footnote{\url{https://github.com/scafi/artifact-2021-ecoop-xc}}~\cite{xc-experiment-with-scafi}, with the latter described in \Cref{chap:state-of-the-art->sec:xc-experiment}.

\section{Protelis} \label{chap:state-of-the-art->sec:protelis}

Protelis is an external domain-specific language derived from the discontinued \textit{Proto}, whose syntax resembles that of C or Java.
%
However, Protelis is purely functional, albeit dynamically typed, and uses a virtual machine written in Java~\cite{protelis} for interpretation.
%
As an external \ac{DSL}, Protelis syntax is close to the \ac{FC} language it implements, which distinguishes it from internal \acp{DSL} like ScaFi and FCPP.
%
As a result, domain branching in Protelis is more transparent in its conditional control statements, such as the \texttt{if} statement, while, in internal DSLs, custom operators must be used to avoid conflicts with the host language's homonymous constructs. 
%
Further information on domain branching can be found in \Cref{chap:background->sec:xc->subsec:alignment}.
%
Nevertheless, the Protelis environment comes with some costs, such as a lack of compiler support for type checking, as it uses duck typing.
%
Additionally, IDE support is exclusively available for the Eclipse platform, as Protelis is based on the Xtext framework~\cite{xtext}.
%
Moreover, external \acp{DSL} like Protelis cannot benefit from the community of developers and libraries of a general-purpose language such as Scala or C++, which are the host languages for ScaFi and FCPP, respectively.

An example of a gradient distance written using Protelis can be found in \Cref{lst:gradient-distance-protelis}.

\lstinputlisting[float, language=Protelis, caption={Gradient distance from a source in Protelis.}, label={lst:gradient-distance-protelis}]{listings/protelis-gradient-distance.pt}


\section{FCPP} \label{chap:state-of-the-art->sec:fcpp}

FCPP is an internal domain-specific language that is written in C++ and is designed for achieving high efficiency and performance for devices with limited resources like microcontrollers and embedded systems~\cite{fcpp}.
%
Although FCPP was originally based on \ac{FC}, it has been updated to support \ac{XC}.
%
As stated in the paper, FCPP suffers more limitations than ScaFi when it comes to avoiding conflicts with the host language, resulting in a less \quotes{clean} syntax~\cite{fcpp}.
%
Additionally, it lacks integration with the Java environment, which is in turn natively supported by ScaFi, being written in Scala.
%
Another critical difference between the design of FCPP and ScaFi is the presence of explicit \texttt{field} types, which are absent in ScaFi thanks to its design around \texttt{foldhood} operations, as described in \Cref{chap:state-of-the-art->sec:scafi}.

\Cref{lst:gradient-distance-fcpp} represents an example of how gradient distance can be written using FCPP.

\lstinputlisting[float, language=C++, caption={Gradient distance from a source in FCPP.}, label={lst:gradient-distance-fcpp}]{listings/fcpp-gradient-distance.cpp}


\section{ScaFi} \label{chap:state-of-the-art->sec:scafi}

\textit{ScaFi}, short for \textit{Sca}la \textit{Fi}elds, is a framework for aggregate programming featuring an internal \ac{DSL} written in pure Scala 2~\cite{scafi} and implementing a variant of the \ac{HFC}.
%
Besides the \ac{DSL}, which represents the core of ScaFi, the framework offers additional components for the simulation, visualization, and deployment of aggregate programs.
%
Scafi cross-compiles for Scala 2.11, 2.12, and 2.13, while its \texttt{core} and \texttt{simulator} packages are also cross-built for \ac{JS} using \textit{Scala.js}~\cite{scala-js}.

The most notable aspect of its \ac{DSL} is the \textit{foldhood} semantics, which abstracts over the concept of \textit{field} or \textit{neighbouring value}, as shown in \Cref{lst:gradient-distance-scafi}.
%
In that usage example, the \texttt{foldhoodPlus} operator invokes and collects the results of the passed expression for each neighbor, including itself, accumulating all of them into a single value using \texttt{Math.min}.
%
With this approach, \texttt{nbr}, which is the main communication primitive of \ac{FC}, can be utilized seamlessly in combination with local values, eliminating the requirement for a \texttt{lift} operator that would typically be necessary to operate with \texttt{field} types, such as in FCPP or in the \quotes{XC: Scala DSL Implementation} experiment, as described in \Cref{chap:state-of-the-art->sec:xc-experiment}.
%
As a limitation of the approach, \texttt{nbr} cannot be used outside one of the \texttt{foldhood} variants.

\lstinputlisting[float, language=Scala, caption={Gradient distance from a source in ScaFi.}, label={lst:gradient-distance-scafi}]{listings/scafi-gradient-distance.scala}

A more thorough examination of ScaFi can be found in \Cref{chap:analysis->sec:scafi-analysis}.

\section{XC: Scala DSL Implementation} \label{chap:state-of-the-art->sec:xc-experiment}

The first implementation of \ac{XC} in Scala is based on ScaFi and presented in the \ac{XC} papers~\cite{xc}~\cite{xc-experiment-with-scafi}.
%
This implementation uses Scala 2.
%
Although ScaFi hides the \texttt{field} abstraction from the user, \textit{NValue}s had to be explicitly implemented in this experiment, given the new semantics they provide.
%
The Scala 2 implicit conversions allowed for the implementation of an automatic conversion from local values to NValues, as explained in the \ac{XC} paper and \Cref{chap:background->sec:xc->subsec:nvalues}.
%
In the experiment, publicly available on GitHub\footnote{\url{https://github.com/scafi/artifact-2021-ecoop-xc}} under the Apache 2.0 License and on Zenodo~\cite{xc-experiment-with-scafi-code}, the \ac{FC} constructs have been implemented using \texttt{exchange}, the only communication primitive of \ac{XC}, suggesting a new syntax for a pure Scala \ac{XC}, later taken as inspiration for \this.

An example of a gradient distance written with this DSL can be found in \Cref{lst:gradient-distance-xc-scala2-dsl}.

\lstinputlisting[float, language=Scala, caption={Gradient distance from a source in XC Scala 2 DSL.}, label={lst:gradient-distance-xc-scala2-dsl}]{listings/xc-experiment-scafi-gradient-distance.scala}

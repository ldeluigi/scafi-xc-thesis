% ! TeX root = ../thesis-main.tex
\begin{abstract}	
% Max 2000 characters, strict.
% Abstract should briefly point out the Context, Problem/Objectives, Methods/Contribution, Results, Conclusions
In the field of macroprogramming, one of the prominent engineering techniques is aggregate computing, which allows the definition of the overall behavior of a network of devices or agents through a single program, called the aggregate program.
%
ScaFi is an aggregate programming framework that comprises an internal Domain Specific Language (DSL) written in Scala 2 with supporting components for simulation, visualization, and execution of aggregate systems, based on the Field Calculus formal language and computational model.
%
Recently, a new formal language and computational model called Exchange Calculus has been proposed, extending the expressiveness of Field Calculus while simplifying its set of primitive constructs.
%
In the meantime, Scala 2 has been succeeded by Scala 3, which introduces a relevant set of new features and improvements over its predecessor.
%
These two novelties have promoted the development of a new ScaFi, called ScaFi-XC, entirely redesigned to be formally based on Exchange Calculus and to make the best use of its new host language, Scala 3, enhancing the original ScaFi with a more expressive and flexible DSL.
%
This thesis presents the design and implementation of ScaFi-XC, focusing on the core DSL and associated components such as the engine and the simulator.
%
In conclusion, ScaFi-XC has been successfully implemented and tested, delivering the expected results and fulfilling the requirements of the stakeholders, which are mainly the researchers and developers of the original ScaFi.
%
Nevertheless, there is still work to be done to have a complete and stable version of the framework, such as porting all the support modules and improving the simulator.
\end{abstract}

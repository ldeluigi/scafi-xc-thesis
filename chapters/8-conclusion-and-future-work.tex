% ! TeX root = ../thesis-main.tex
%----------------------------------------------------------------------------------------
\chapter{Conclusion and Future Work}
\label{chap:conclusion-and-future-work}
%----------------------------------------------------------------------------------------

The objective of this work has been to develop a new \ac{DSL} foundation, which would serve as the basis for a new framework named \this.
%
The framework proposed is intended to redesign the ScaFi toolkit, is implemented using the Scala 3 programming language, and is supported by the Exchange Calculus computational model and formal language.
%
The development of \this involved prototyping, interviews, and the application of advanced programming patterns with Scala 3.
%
Continuous integration and acceptance testing were utilized to ensure the quality and reliability of the framework.
%
The requirements for \this, collected through interviews with stakeholders, have all been satisfied, including optional ones, so the final result can be considered a success.
%
However, the current implementation covers only a fraction of the functionality offered by the original ScaFi framework, leaving room for further extensions.

The concrete implementations presented here aim for simplicity, readability, correctness, and reusability, deferring concerns about performance and efficiency for future iterations.
%
Furthermore, the developed simulator offers a very restricted set of features in comparison to the original ScaFi simulator, needing improvements to support more complex scenarios and one or more graphical interfaces for the user to interact with the simulator.
%
The following paragraphs provide a list of possible future developments.

\paragraph{Performance optimization} The current implementation of the context and the simulator is not optimized for performance.
%
For example, the context leverages the \texttt{Map} data structure to represent \texttt{ValueTree}s, which is not the most efficient data structure for this purpose, as it stores multiple copies of all the common prefixes of the keys.

\paragraph{Complete re-implementation of the core module} The \texttt{core} module is the foundation of the ScaFi framework, as it contains the basic building blocks for the development of aggregate programs.
%
In \this, it has been implemented only partially, since the only advanced construct implemented is the gradient.
%
A comprehensive extension to include the full standard library is essential for the framework's completeness.

\paragraph{Enhancements of the simulator} The current simulator does not support many of the features of the original ScaFi simulator and is limited to discrete time.
%
Extensions of the implementation or a complete redesign can provide a more comprehensive feature set, including support for graphical user interfaces.

\paragraph{Support for real-world distributed systems} The original ScaFi allowed the deployment of aggregate programs on real-world distributed systems with the \texttt{spala} and \texttt{distributed} modules, currently absent in \this.
%
Implementing such support is a crucial step to consider \this a valid replacement for the original ScaFi.

\paragraph{Experimental developments with aggregate programming} The reusability and the modularity of the new core module allow it to be extended with new, experimental libraries and semantics, fostering research projects exploring novel aspects of aggregate programming.

\paragraph{Adding more acceptance tests} Most of the reliability of the framework comes from the functionality and readability of its tests.
%
In particular, acceptance tests are designed to be proofreadable by experts in the field, and they are the most important tests for the framework.
%
Nevertheless, only a few acceptance tests are currently present.
%
Strengthening the test suite would enhance the reliability of the framework, ensuring its robustness in a wider range of scenarios.

\paragraph{Improvement of the Alchemist incarnation} The integration with the Alchemist simulator is still a prototype.
%
A more complete implementation would enable more sophisticated simulation scenarios, such as those involving situated agents, with sensors and actuators actively managed by the aggregate program.

\paragraph{Survey evaluation of the framework} A survey evaluation of the framework could be conducted to assess the usability and effectiveness of the framework in the development of aggregate programs.
%
Additionally, it could provide conclusive results regarding the impact of the Context-based constraints on shared values discussed in \Cref{chap:implementation->sec:context-based-constraints}.
%
Whether or not the proposed changes to the core represent a valid improvement is still an open question, and a survey evaluation of the proposal could provide a definitive answer.

In conclusion, while \this marks a significant step towards a new ScaFi framework, there is still much work to be done.
%
With continued development and iteration, it has the potential to significantly contribute to the field of aggregate programming, both as a development framework and as a research tool.

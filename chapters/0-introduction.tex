% ! TeX root = ../thesis-main.tex
%----------------------------------------------------------------------------------------
\chapter{Introduction}
\label{chap:introduction}
%----------------------------------------------------------------------------------------
\acp{CAS} are a particular kind of situated, distributed systems where a collection of individuals, also called agents, exhibits a non-chaotic behavior with \textit{self-*} properties, such as self-organization, self-healing, and self-configuration\cite{macroprogramming-state-of-the-art}.
%
Many of the properties mentioned above cannot be reduced to or derived from an individual perspective on the behavior of one of the agents, but rather are \textit{emergent} from the complex network and dynamic interactions within the system and with the environment\cite{macroprogramming-state-of-the-art}.
%
\acp{CAS} can be considered a subset of \acp{MAS}, and the programming of their behavior as a whole can be referred to as \textit{Macroprogramming}\cite{macroprogramming-state-of-the-art}.

With the expected diffusion of large-scale cyber-physical systems, pushed by the already pervasive \ac{IoT} and edge computing trends\cite{scafi}, \acp{CAS} engineering is 

TODO

%NOTE: What are the fundamental characteristics of the aggregate programming tools ScaFi,
% Protelis, and FCPP?
%How does the Higher-Order Field Calculus (HFC) serve as the theoretical 
%foundation for these tools, and how does it provide theoretical consistency and guarantees for
% aggregate programming?
%What are the differences between external and internal domain-specific languages, as illustrated 
%by Protelis and FCPP, respectively?

Aggregate computing is a macro-programming paradigm where a single program (called the aggregate program) defines the overall behaviour of a network of devices or agents\cite{macroprogramming-state-of-the-art}.

\paragraph{Structure of the Thesis}

TODO
